There were 11,061 stars in common provided by LAMOST which overlapped with APOGEE. These stars were provided as individual files with a flux, variance and wavelength array. 
Of these 11,061 stars in common, a high fidelity sample of stars was selected to comprise the training set (note to Anna: This should be redone to better cover label space with some criteria on the
number of stars as a function of the labels - e.g. so more stars at low metallicity).
This criteria was that the signal to noise of the APOGEE spectra of these stars (1) be SNR > 400 (for high fidelity labels from APOGEE, assuming - labels are higher fidelity for the stars observed at
higher SNR), (2) that the velocity scatter from
APOGEE for these stars was < 1 km/s and (3) that the BAD STAR flag was not set (no flag set in ASPCAPFLAG)
From this criteria 1722 stars were returned as training stars. The model was constructed from the LAMOST spectra of these stars, given APOGEE labels.  
The remaining 9339 stars were used as test objects - to test how the labels output from The Cannon of the LAMOST spectra compared to the APOGEE labels for these stars, from APOGEE spectra. 

For both test and training data, the stars were shifted back to the rest frame by using the redshift value provided in the header of each star by LAMOST, where 
wavelength_shift = redshift*xdata_in 
the data were then resampled on to a common, evenly spaced grid (but this should to be corrected - change spacing to evenly spaced in logarithm space not linear space) of 0.85 Angstroms (which is the spacing at the
lower wavelength limit of the data). Upper and lower cuts were made to the data to span from 3900 to 8800 Angstroms. 
This data, now radial velocity corrected and resampled on to a common grid, was then psuedo-continuum normalised using a running 90th percentile, as the first input to The Cannon in order to determine
the continuum pixels for proper continuum normalisation. 
All operations done to the flux were also performed on the variance arrays. 

